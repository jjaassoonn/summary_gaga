\documentclass{article}
\usepackage[a4paper,margin=0.6in]{geometry}
\usepackage{amsmath}
\usepackage{amssymb}
\usepackage[english]{babel}
\usepackage{amsthm}
\usepackage{tikz-cd}
\usepackage{enumitem}
\usepackage{calc}

\theoremstyle{definition}
\newtheorem{definition}{Definition}
\newtheorem{theorem}{Theorem}
\newtheorem*{notation}{Notation}

\newcommand{\Spec}{\mathrm{Spec}}
\newcommand{\ssheaf}[1]{\widetilde{#1}}
\newcommand{\tensorC}[2]{#1\otimes_\mathbb C #2}
\newcommand{\tensor}[2]{#1\otimes#2}
\newcommand{\so}[1]{(#1, \mathcal O_{#1})}
\newcommand{\id}{\mathrm{id}}
\newcommand{\sr}[1]{(\Spec{#1},\ssheaf{#1})}
\newcommand{\srb}[1]{(\Spec{(#1)},\ssheaf{#1})}

\title{Summary of GAGA}
\date{}
\author{Jujian Zhang}

\begin{document}

\maketitle

\section{Affine schemes of finite type}
\section{$\mathbb P^n$ as a Group Scheme}

\subsection{Product of Affine Schemes}

\begin{notation}
In this subsection, $(X, \mathcal O_X)=(\Spec R,\ssheaf R)$ and $(Y,\mathcal O_Y)=(\Spec S, \ssheaf S)$ are affine schemes of finite type over $\mathbb C$ where $R$ and $S$ are finitely generated $\mathbb C$-algebra.
\end{notation}

\begin{definition}
	$\so X\times \so Y$ is defined as $(\Spec(\tensorC R S),\ssheaf{\tensorC R S})$.
\end{definition}

\paragraph{} $\so X\times \so Y$ satisfies the universal property of product in the category of affine schemes of finite type over $\mathbb C$. The natural inclusions $i_1:R\to \tensorC R S$ given by $i_1(r)=\tensor r 1$ and $i_2:S\to \tensorC R S$ given by $i_2(s)=\tensor 1 s$ induces $\so{X}\times\so Y\to \so X$ and $\so{X}\times\so Y\to \so Y$; and for any $\so Z$, an affine schemes of finite type over $\mathbb C$, then any maps $\so Z \to \so X$ and $\so Z\to\so Y$ factors uniquely through $\so{X\times Y}$, i.e. we have the following commutative diagram
\[
\begin{tikzcd}
 & & \so X\\
\so Z \arrow[rru, bend left=10] \arrow[rrd, bend right=10] \arrow[r, dashed, "\exists!" description] & \so{X}\times\so Y \arrow[ru] \arrow[rd]\\
 & & \so Y
\end{tikzcd}.
\]

\subsection{Affine Group Scheme}
\begin{notation}
In this subsection, $\so G=\sr R$ is an affine scheme of finite type over $\mathbb C$.	
\end{notation}

\begin{definition}
$\so G$ is an affine group scheme if there are $\mu: \so G\times\so G\to\so G$, $e:(\Spec\mathbb C,\widetilde{\mathbb C})\to\so G$ and $\iota:\so G\to\so G$ such that the following diagrams commute
\begin{itemize}
\item associativity of $\mu$:
\begin{equation*}
\begin{tikzcd}
\so G\times\so G\times\so G \arrow[r,"{\id\times\mu}"] \arrow[d,"\mu\times\id"] & 
\so G\times\so G\arrow[d, "\mu"] \\
\so G\times\so G \arrow[r, "\mu"] & \so G
\end{tikzcd}
\end{equation*}


\item neutral element $e$:
\begin{equation*}
\begin{tikzcd}
\sr{\mathbb C}\times\so G\times\sr{\mathbb C}\arrow[rr,"{\id\times\id\times e}"]\arrow[rd, "{\cong}" description]
\arrow[dd,"{e\times\id\times\id}"] 
& & 
\sr{\mathbb C}\times\so G\times\so G
\arrow[d,"\cong"] \\
& \so G \arrow[rd, "\id"]& \so G\times\so G\arrow[d, "\mu"] \\
\so G\times\so G\times\sr{\mathbb C}\arrow[r,"\cong"] & \so G\times\so G\arrow[r, "\mu"] &\so G
\end{tikzcd},
\end{equation*}
% where the diagonal arrow signifies that the composition from top left to bottom right are both equal to identity on $\so G$ after identifying $\sr{\mathbb{C}}\times\so G\times\sr{\mathbb C}$ with $\so G$

\item inverse $i$:
\begin{equation*}
\begin{tikzcd}
    \so G\arrow[rr,"{(\id,\iota)}"]
    \arrow[rd, "\text{constant}" description] \arrow[dd, "{(\iota, \id)}"] & & \so G\times \so G \arrow[dd, "\cdot"] \\
	& \sr{\mathbb C} \arrow[rd, "e"] & \\
\so G\times\so G\arrow[rr, "\mu"] & & \so G
\end{tikzcd}
\end{equation*}

\end{itemize}

\begin{notation}
$\mu$ is often written as $\cdot$ to indicate multiplication and $\iota$ as $^{-1}$ to indicate inverse.
\end{notation}
\label{def:group-scheme}
\end{definition}

\paragraph{} In definition \ref{def:group-scheme}, since $\so G\times\so G=\srb{\tensorC R R}$, the maps $\mu$ ($e$, $\iota$ resp.) gives a map $\mu':R\to\tensorC R R$ ($e' : R\to \mathbb C$, $\iota':\mathbb R\to\mathbb R$ resp.) such that the following diagrams commute

\noindent%
\begin{minipage}{0.25\linewidth}
\begin{equation*}
\begin{tikzcd}
R \arrow[r, "\mu'"] \arrow[d, "\mu'"]& \tensorC R R \arrow[d, "\tensor\id\mu'"]\\
\tensorC R R \arrow[r, "\tensor{\mu'}{\id}"] & \tensorC R{\tensorC R R}
\end{tikzcd}    
\end{equation*}
\end{minipage}%
\begin{minipage}{0.5\linewidth}
\begin{equation*}
\begin{tikzcd}
R \arrow[r, "\mu'"]\arrow[d, "\mu'"]\arrow[rd,"\id"] & \tensorC R R \arrow[r, "\cong"] & \tensorC R {\tensorC R {\mathbb C}}\arrow[dd,"{\tensor{e'}{\tensor\id\id}}"]\\
\tensorC R R\arrow[d, "\cong"] & R\arrow[rd, "\cong"] \\
\tensorC{\mathbb C}{\tensorC R R}\arrow[rr, "{\tensor\id{\tensor\id{e'}}}"] & & \tensorC{\mathbb C}{\tensorC R \mathbb C}
\end{tikzcd}
\end{equation*}
\end{minipage}%
\begin{minipage}{0.25\linewidth}
\begin{equation*}
\begin{tikzcd}
R \arrow[rr, "\mu'"]\arrow[dd,"\mu'"]\arrow[rd, "e'"]& & \tensorC R R \arrow[dd, "\tensor{r}{r'}\mapsto\iota'(r)r'" description]\\
& \mathbb C\arrow[rd, "*\text{see below}" description] & \\
\tensorC R R \arrow[rr, "\tensor{r}{r'}\mapsto r\iota'(r')" description]& & R
\end{tikzcd}
\end{equation*}
\end{minipage},
where (*) is the map $\mathbb C\to R$ giving $R$ the $\mathbb C$-algebra structure.

\subsubsection{Trivial Group Scheme}
Consider $\mathbb C$ as the trivial $\mathbb C$-algebra. Let $\mu': \mathbb C\to\tensorC{\mathbb C}{\mathbb C}$ be $\id:\mathbb C\to\mathbb C$ after identifying $\tensorC{\mathbb C}{\mathbb C}\cong\mathbb C$ and $e',\iota':\mathbb C\to\mathbb C$ all be identity on $\mathbb C$. In this case $\so G=\sr{\mathbb C}$ is a space with only one point.

\subsubsection{A Non-trivial Group Scheme}
\label{sec:hopf-algebra}
\begin{notation}
In this subsubsection, $R$ is $\mathbb C[x]$ and localization maps are denoted by $\alpha$ with subscript, for example $\alpha_x : R\to R\left[\frac1x\right]$ localises $x$.
\end{notation}
\paragraph{} Consider the ring of Laurent polynomial $\mathbb C[x,x^{-1}]\cong R\left[\frac1x\right]$ as a $\mathbb C$-algebra. Let the map $\nu':R\to\tensorC R R$ be given by $x\mapsto \tensor x x$. Then 
\begin{equation*}
\begin{tikzcd}
R\arrow[r, "\nu'"] & \tensorC R R \arrow[r, "\tensor\phi\phi", "\phi\text{ is the inclusion}"'] %& \tensorC{\mathbb C[x]}{\mathbb C[y]}  \arrow[r, "\tensor\phi\phi", "\phi\text{ is the inclusion}"']
&[5em] \tensorC{\mathbb C[x,x^{-1}]}{\mathbb C[x, x^{-1}]}
\end{tikzcd}
\end{equation*}
send $x$ to $\tensor x x$ which is invertible, hence by universal property of localisation, there is a unique map $\mu'$ such that
\begin{equation*}
\begin{tikzcd}
R\arrow[r, "\nu'"]\arrow[d, "\alpha_x"] & \tensorC R R\arrow[d, "\tensor \phi \phi"] & \\
 % & \tensorC{\mathbb C[x]}{\mathbb C[y]}\arrow[d, "\tensor{\phi}{\phi}"] & \\
R\left[\frac1x\right]\arrow[r, dashed, "\exists!"]\arrow[d, "\cong"] & \tensorC{\mathbb C[x,x^{-1}]}{\mathbb C[x, x^{-1}]} \\
\mathbb C[x, x^{-1}]\arrow[ru, dashed, "\mu'"']
\end{tikzcd}.
\end{equation*}
The map $e':\mathbb C[x,x^{-1}]\to\mathbb C[x,x^{-1}]$ is given by evaluating at $x=1$ and $\iota': \mathbb C[x,x^{-1}]\to\mathbb C[x,x^{-1}]$ is given by $\iota'(x)=x^{-1}$ and $\iota'(x^{-1})=x$. 

\paragraph{Associativity of $\mu'$} Need to check $x\in\mathbb C[x,x^{-1}]$ and $x^{-1}\in\mathbb C[x,x^{-1}]$. For $x$, the first composition
$$
\begin{tikzcd}
{x \in \mathbb C[x, x^{-1}]} \arrow[r, mapsto, "\mu'"] & {\tensor x x} \arrow[d, mapsto, "{\tensor\id{\mu'}}"] \\
& {\tensor{\tensor x x}{x}}
\end{tikzcd}
$$ equals to the second composition
$$
\begin{tikzcd}
{x\in\mathbb C[x,x^{-1}]}\arrow[d,mapsto, "\mu'"] \\
{\tensor x x}\arrow[r, mapsto, "\tensor{\mu'}{\id}"] & \tensor{x}{\tensor x x}.
\end{tikzcd}
$$ For $x^{-1}$, just replace every $x$ by $x^{-1}$, and get the same thing.

\paragraph{Neutral element $e'$} For $x\in\mathbb C[x, x^{-1}]$, the first composition is
$$
\begin{tikzcd}
x \arrow[r, mapsto, "\mu'"] & {\tensor x x} \arrow[r, "\cong"] & \tensor{x}{\tensor x 1}\arrow[dd, mapsto, "{\tensor{e'}{\tensor \id \id}}"] \\
\\
& & \tensor1{\tensor x 1};
\end{tikzcd}
$$ the second composition is
$$
\begin{tikzcd}
x \arrow[d, mapsto, "\mu'"] \\
{\tensor x x}\arrow[d, mapsto, "\cong"] \\
\tensor1{\tensor x x}\arrow[rr,mapsto, "\tensor{\id}{\tensor\id{e'}}"] & & \tensor1{\tensor x 1};
\end{tikzcd}
$$ and the diagonal composition is
$
\begin{tikzcd}
x\arrow[r, mapsto, "\id"]
& x\arrow[r, mapsto, "\cong"]
& \tensor1{\tensor x 1}.
\end{tikzcd}
$ The case for $x^{-1}$ is the same as above because $x^{-1}$ evaluate at $1$ is also $1$.

\paragraph{Inverse $\iota'$} The first composition is
$$
\begin{tikzcd}
  x \arrow[r, mapsto, "\mu'"] & {\tensor x x} \arrow[d, mapsto, "{\tensor{r}{r'}\mapsto\iota'(r)r'}"]\\
  & {x^{-1}x=1}, 
\end{tikzcd}
$$ and the second composition is
$$
\begin{tikzcd}[column sep=huge]
  x \arrow[d, "\mu'"] \\
  {\tensor x x} \arrow[r, "{\tensor{r}{r'}\mapsto r\iota'(r')}"] & x x^{-1}=1,
\end{tikzcd}
$$ and the third composition is $
\begin{tikzcd}
  x \arrow[r, "e'"] & 1 \arrow[r] & 1
\end{tikzcd}
$. The case for $x^{-1}$ is similar.

\subsection{Affine group scheme acting on affine group scheme}
\begin{notation}
In this subsection, assume $\so G = \sr R$, $\so X=\sr S$ and $\so Y=\srb{S'}$ are affine schemes finite type over $\mathbb C$ and that $\so G$ is a group scheme.
\end{notation}

\begin{definition}
  An action of $\so G$ on $\so X$ is given by a homomorphism $a' : S \to \tensorC R S$ such that
\newline
\noindent  \begin{minipage}{0.45\linewidth}
    \begin{equation*}
      \begin{tikzcd}
        S \arrow[r, "a'"]\arrow[d, "a'"] & \tensorC R S\arrow[d, "{\tensor\id{a'}}"] \\
        \tensorC R S \arrow[r, "{\tensor{\mu'}\id}"] & \tensorC R{\tensorC R S},
      \end{tikzcd}
    \end{equation*}
  \end{minipage} and
  \noindent\begin{minipage}{0.45\linewidth}
    \begin{equation*}
      \begin{tikzcd}
        & \tensorC R S \arrow[d, "{\tensor{e'}\id}"]\\
        S \arrow[ru, "a'"] \arrow[rd, "\id"] & \tensorC{\mathbb C} S\arrow[d, no head, "\cong"]\\
        & S
      \end{tikzcd}
    \end{equation*}
  \end{minipage}
\end{definition}

\begin{definition}\label{def:G-morphism}
  Suppose $\so G$ acts on $\so X$ and $\so Y$. A morphism $\beta : \so X \to \so Y$ of schemes of finite type over $\mathbb C$ is called a $G$-morphism if the following diagram commutes
  \begin{equation*}
    \begin{tikzcd}
      \so G\times \so X \arrow[r, "a_X"] \arrow[d, "{\id\times\beta}"]& \so X \arrow[d, "\beta"]\\
      \so G\times \so Y \arrow[r, "a_Y"]& \so Y. \\
    \end{tikzcd}
  \end{equation*}
\end{definition}

We can rephrase definition \ref{def:G-morphism} in terms of ring/algebra homomorphism:
$\beta$ is $\srb{\beta'}$ for a unique $\mathbb C$-algebra homomorphism $\beta':S'\to S$, %TODO : addjustification and reference
the commutative squre becomes
\begin{equation*}
\begin{tikzcd}
  S' \arrow[r, "{a_{S'}}"]\arrow[d, "\beta'"]& \tensorC{R}{S'} \arrow[d, "{\tensor{\id}{\beta'}}"]\\
  S \arrow[r, "{a_S}"]& \tensorC{R}{S}.
\end{tikzcd}
\end{equation*}

\begin{notation}
We say that $S$ is a $G$-ring if the group scheme $\so G$ acts on $\sr S$. And $\beta'$ as above is a $G$-homomorphism of $G$-rings.
\end{notation}

\begin{theorem}
Let $S$ be a $G$-ring and $f\in S$ such that $f\ne 0$ and $a'(f)=\tensor r f$. Then $r\in R$ is invertible.
\end{theorem}
\begin{proof}
  Since $\so G=\sr R$ is a group scheme, we have the following commutative diagram:
  \begin{equation*}
    \begin{tikzcd}[column sep=huge]
      R \arrow[r, "e'"]\arrow[d, "\mu'"]& \mathbb C\arrow[d, "\rho"] \\
      {\tensorC R R} \arrow[r, "\tensor r{r'}\mapsto r\iota'(r')"] & R,
    \end{tikzcd}
  \end{equation*}
  where $\rho : \mathbb C\to R$ is the algebra map. We tensor this diagram with $S$:
    \begin{equation*}
    \begin{tikzcd}[column sep=7em]
      \tensorC R S \arrow[r, "{\tensor{e'}\id}"]\arrow[d, "\tensor\mu\id'"]& \tensorC{\mathbb C}S\arrow[d, "{\tensor\rho\id}"] \\
      {\tensorC{\tensorC R R}S} \arrow[r, "\tensor{(\tensor r{r'}\mapsto r\iota'(r'))}\id"] & \tensorC R S,
    \end{tikzcd}
  \end{equation*}
  so by defniition of action, we get the following commutative diagram:
  \begin{equation*}
    \begin{tikzcd}[column sep=7em]
        &  S \arrow[rd, no head, "\cong"] \\
      S\arrow[ru, "\id"] \arrow[r, "a'"]\arrow[d, "a'"] & \tensorC R S \arrow[r, "{\tensor{e'}\id}"]\arrow[d, "\tensor\mu\id'"]& \tensorC{\mathbb C}S\arrow[d, "{\tensor\rho\id}"] \\
      {\tensorC R S} \arrow[r, "\tensor\id{a'}"]& {\tensorC{\tensorC R R}S} \arrow[r, "\tensor{(\tensor r{r'}\mapsto r\iota'(r'))}\id"] & \tensorC R S.
    \end{tikzcd}
  \end{equation*}
  So for $f\in S$, the first composition from top left to bottom right is equal to
$
    \begin{tikzcd}
      f \arrow[d, mapsto]\\
      \tensor r f\arrow[r, mapsto] & \tensor{\tensor r r} f\arrow[r, mapsto] & \tensor{r\cdot\iota'(r)}f,
    \end{tikzcd}
  $
  and the second composition is equal to
$
    \begin{tikzcd}
      & f \arrow[rd, mapsto]& \\
      f \arrow[ru, mapsto, "\id"] \arrow[r, mapsto]& \tensor r f & \tensor1f \arrow[d, mapsto]\\
      & & \tensor1f,
    \end{tikzcd}
 $
  hence $\tensor{r\cdot\iota'(r)}f=\tensor1f$, since $f\ne0$, $r\cdot\iota'(r)=1$, i.e. $r$ is invertible in $R$.
\end{proof}

\subsubsection{Example of action by group scheme}
We consider the example in section \ref{sec:hopf-algebra}. So $R = \mathbb C[t, t^{-1}]$ and $S=\mathbb C[x_0, x_1,\cdots,x_n]$. Consider $\sr R$ acting on $\sr S$ by $a'(x_i)=\tensor{t^{-1}}{x_i}$. Check commutativity:
$
\begin{tikzcd}
  {x_i}\arrow[r, mapsto, "a'"] & {\tensor{t^{-1}}{x_i}} \arrow[d, "{\tensor\id{a'}}"] \\
  & {\tensor{t^{-1}}{\tensor{t^{-1}}{x_i}}}
\end{tikzcd}
$and $
\begin{tikzcd}
  {x_i}\arrow[d, mapsto, "a'"] \\
  {\tensor{t^{-1}}{x_i}}\arrow[r, mapsto, "{\tensor{\mu'}\id}"] & {\tensor{t^{-1}}{\tensor{t^{-1}}{x_i}}}
\end{tikzcd}
$; $
\begin{tikzcd}
  & {\tensor{t^{-1}}{x_i}}\arrow[d, mapsto, "\tensor{e'}\id"] \\
  {x_i} \arrow[ru, "a'"] & {\tensor1{x_i}} \arrow[d]\\
  & {x_i}
\end{tikzcd}
$ obvisouly commutes with $\id$.

\subsubsection{Examples of $G$-homomorphism}

Suppose $S$ is a $G$-ring and $f\in S$ is such that $a'_S(f)=\tensor r f$. So $r\in R$ is invertible. Hence the composite
$
\begin{tikzcd}
  S \arrow[r, "a_S'"] & \tensorC R S \arrow[r, "\tensor\id{\alpha_f}"] & \tensorC R {S\left[\frac1f\right]}
\end{tikzcd}
$ takes $f\in S$ to an invertible $\tensor r f\in\tensorC R{S\left[\frac1f\right]}$. Thus this composite
factors through $\alpha_f:S\to S\left[\frac1f\right]$, i.e.
$
\begin{tikzcd}
  S \arrow[r, "a'_S"]\arrow[d, "\alpha_f"]& \tensorC R S\arrow[d, "\tensor\id{\alpha_f}"] \\
  {S\left[\frac1f\right]}\arrow[r, dashed, "\exists a'_{S\left[\frac1f\right]}"', description] & \tensorC R{S\left[\frac1f\right]}
\end{tikzcd}.
$

$\begin{tikzcd}
  S \arrow[r, "a'_S"]\arrow[d, "\alpha_f"]& \tensorC R S\arrow[d, "\tensor\id{\alpha_f}"] \arrow[r, "\tensor\id{a'_S}"]& \tensorC{\tensorC R R} S\arrow[d, "\tensor{\tensor\id\id}{\alpha_f}"]\\
  {S\left[\frac1f\right]}\arrow[r, dashed, "\exists a'_{S\left[\frac1f\right]}"', description] & \tensorC R{S\left[\frac1f\right]} \arrow[r, "\tensor{\mu'}{\id}"]& \tensorC{\tensorC R R}{S\left[\frac1f\right]}
\end{tikzcd}$ is also commutative because $S$ is a $G$-ring. 

Temporarily denote the composite as $c$, then $a'_{S\left[\frac1f\right]}(\frac g{f^n})=c(f)^{-n}c(g)=((\tensor\id{\alpha_f})(\tensor r f))^{-n}c(g)=
\left(\tensor rf\right)^{-n}c(g)=\left(\tensor{r^{-n}}{\frac1{f^n}}\right)c(g)$.
Then the following holds:
\begin{enumerate}
\item $S\left[\frac1f\right]$ is a $G$-ring.
  \begin{proof}
    check commutativity of
$
      \begin{tikzcd}
        {S\left[\frac1f\right]}\arrow[r, "a'"]\arrow[d, "a'"] & \tensorC R {S\left[\frac1f\right]} \arrow[d, "\tensor\id{a'}"]\\
        {\tensorC R{S\left[\frac1f\right]}}\arrow[r, "\tensor{\mu'}\id"] & \tensorC{\tensorC R R}{S\left[\frac1f\right]}
      \end{tikzcd}
$: we want $(\tensor\id{a'})\left(\left(\tensor{r^{-n}}{\frac1{f^n}}\right)c(g)\right)=(\tensor{\mu'}\id)\left(\left(\tensor{r^{-n}}{\frac1{f^n}}\right)c(g)\right)$. Left hand side is $(\tensor\id{a'})(\tensor{r^{-n}}{\frac1{f^n}})\cdot(\tensor\id{a'})(c(g))=\left(\tensor{r^{-n}}{\tensor{r^{-n}}{\frac1{f^n}}}\right)(\tensor\id{a'})(c(g))$
  \end{proof}
\end{enumerate}
\end{document}
